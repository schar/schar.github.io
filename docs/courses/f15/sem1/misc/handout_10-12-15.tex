\documentclass{article}
\synctex=1
\usepackage{stmaryrd,mathtools,amsmath,amssymb,booktabs,parskip}
\usepackage[silent]{mathspec}
\defaultfontfeatures{Ligatures=TeX}
\setmainfont[%
  Numbers=OldStyle,
  Kerning=Uppercase,
  Scale=MatchLowercase
]{Aldus nova Pro}
\setsansfont[%
  BoldFont={Univers Next Pro Medium},
  Scale=.85
]{Univers Next Pro Regular} 
\setmonofont[%
  Scale=MatchLowercase
]{Andale Mono}
\setmathfont(Digits,Latin,Greek)[%
  Scale=MatchLowercase
]{Neo Euler}
\scriptspace=0pt 
\usepackage{MnSymbol}
\usepackage[normalem]{ulem}
\usepackage[colorlinks]{hyperref}
\newcommand{\sv}[1]{\ensuremath{\lsem #1 \rsem}}
\newcommand{\ceq}{\coloneqq}
\newcommand{\set}[1]{\ensuremath{\{#1\}}}
\newcommand{\Ra}{\Rightarrow}
\newcommand{\ra}{\rightarrow}
\newcommand{\con}[1]{\textsc{\textsf{#1}}}
\usepackage[margin=1.35in]{geometry}
\setlength{\parindent}{0pt}
\author{Simon Charlow
(\href{mailto:simon.charlow@rutgers.edu}{{simon.charlow@rutgers.edu}})}
\title{\textbf{First-order predicate logic (FOPL)}}
\begin{document}\maketitle

\section{Introducing FOPL by example}

We begin with statements of predication (and statements built from statements of
predication):
\begin{enumerate}
  \item
    Bob is human
    \\
    $\emph{human}\,\emph{bob}$

  \item
    $z$ is French and not German
    \\
    $\emph{french}\,z \wedge \neg\emph{german}\,z$
\end{enumerate}

And we can add relations to the mix:
\begin{enumerate}
  \item
    Mary saw John
    \\
    $\emph{saw}(\emph{mary}, \emph{john})$

  \item
    if $x$ is French, then $x$ didn't show $y$ to Steve
    \\
    $\emph{french}\,x \Ra \neg\emph{show}(x, y, \emph{steve})$
\end{enumerate}

We also have quantified statements:
\begin{enumerate}
  \item
    Something stinks
    \\
    $\exists x.\,\emph{stinks}\,x$

  \item
    $z$ likes everything
    \\
    $\forall y.\, \emph{likes}(z, y)$

  \item
    Everyone is French and German
    \\
    $\forall z.\,\emph{french}\,z \wedge \emph{german}\,z$
\end{enumerate}

Restricted quantification is built from $\exists$ and $\wedge$, or $\forall$ and
$\Ra$ (Exercise: Why \emph{these} pairings? Why not $\exists$ and $\Ra$, or
$\forall$ and $\wedge$?):%
\footnote{%
\label{fn}%
  I omit some parentheses because $a \wedge (b \wedge c)$ is equivalent to $(a
  \wedge b) \wedge c$. Also, similarly to the $\lambda$ calculus, we will always
  assume that the scope of $\exists \nu$ and $\forall \nu$ extend as far to the
  right as possible.
}
\begin{enumerate}
  \item
    A cat meowed
    \\
    $\exists x.\,\emph{cat}\,x \wedge \emph{meowed}\,x$

  \item
    Every cat meowed
    \\
    $\forall z.\, \emph{cat}\,z \Ra \emph{meowed}\,z$

  \item
    A linguist likes a philosopher
    \\
    $\exists x.\,\emph{linguist}\,x \wedge \exists y.\, \emph{philosopher}\,y
    \wedge \emph{likes}(x,y)$
    \\
    $\exists x.\,\exists y.\,\emph{linguist}\,x \wedge \emph{philosopher}\,y
    \wedge \emph{likes}(x,y)$    
\end{enumerate}

We can also represent some cardinality statements:
\begin{enumerate}
  \item
    At least two things are blue
    \\
    $\exists x.\,\exists y.\,x \neq y \wedge \emph{blue}\,x \wedge
    \emph{blue}\,y$

  \item
    At least three boys are French
    \\
    $\exists x.\,\emph{boy}\,x \wedge \exists y.\, \emph{boy}\,y \wedge \exists z.\,
    \emph{boy}\,z \wedge x \neq y \wedge y \neq z \wedge x \neq z \wedge
    \emph{french}\,x \wedge \emph{french}\,y \wedge \emph{french}\,z$
\end{enumerate}

\section{Negation and duality}

Quantified statements can be negated:
\begin{enumerate}
  \item
    Nothing is blue (i.e.~it's false something is blue)
    \\
    $\neg\exists x.\, \emph{blue}\,x$

  \item
    Not every linguist came:
    \\
    $\neg\forall x.\,\emph{linguist}\,x \Ra \emph{came}\,x$
\end{enumerate}

And just as $\vee$ and $\wedge$ are DeMorgan duals with respect to negation\dots
\begin{enumerate}
  \item
    $\neg(\phi \vee \psi) = \neg \phi \wedge \neg \psi$

  \item
    $\neg(\phi \wedge \psi) = \neg \phi \vee \neg \psi$
\end{enumerate}

\dots\ So are $\exists$ and $\forall$:
\begin{enumerate}
  \item
    $\neg\exists \nu.\,\phi = \forall \nu.\,\neg\phi$

  \item
    $\neg\forall \nu.\,\phi = \exists \nu.\,\neg\phi$    
\end{enumerate}

So we can represent our first two negated statements equivalently as follows:
\begin{enumerate}
  \item
    Nothing is blue (i.e.~it's false something is blue)
    \\
    $\neg\exists x.\, \emph{blue}\,x = \forall x.\,\neg\emph{blue}\,x$

  \item
    Not every linguist came:
    \\
    $\neg\forall x.\,\emph{linguist}\,x \Ra \emph{came}\,x = \exists
    x.\,\neg(\emph{linguist}\,x \Ra \emph{came}\,x)$
    \\
    $= \exists x.\, \emph{linguist}\,x \wedge \neg \emph{came}\,x$ \hfill [since
    $\neg(\phi \Ra \psi) = \neg(\neg\phi \vee \psi) = \phi \wedge \neg \psi$]
\end{enumerate}

\section{Scope ambiguity}

Representing scope ambiguity of the quantifiers and negation:
\begin{enumerate}
  \item
    John didn't see a famous linguist
    \\
    $\neg\exists x.\,\emph{famous}\, x \wedge \emph{linguist}\,x \wedge
    \emph{see} (\emph{john},\,x)$
    \\
    $\exists x.\,\emph{famous}\, x \wedge \emph{linguist}\,x \wedge \neg
    \emph{see} (\emph{john}, \,x)$
    
  \item
    Every boy isn't french
    \\
    $\forall x.\, \emph{boy}\,x \Ra \neg\emph{french}\,x$
    \\
    $\neg\forall x.\, \emph{boy}\,x \Ra \emph{french}\,x$
\end{enumerate}

(Some of these representations could be given equivalently by appealing to
duality w.r.t.~negation.)

Notice the placement of the negation with respect to the restriction. What's
wrong with the following?
\[\exists x.\, \neg(\emph{famous}\,x \wedge \emph{linguist}\,x \wedge
\emph{see}(\emph{john}, x))\]

Representing scope ambiguities with two quantifiers
\begin{enumerate}
  \item
    A linguist met every philosopher
    \\
    $\exists x.\, \emph{linguist}\,x \wedge (\forall y.\, \emph{philosopher}\,y
    \Ra \emph{met}(x, y))$
    \\
    $\forall y.\, \emph{philosopher}\,y \Ra (\exists x.\, \emph{linguist}\,x
  \wedge \emph{met}(x, y))$
\end{enumerate}

Each of these cases has a \emph{surface-scope} interpretation, on which the
order of the operators corresponds to their linear order in the English
sentence, and an \emph{inverse-scope} interpretation, on which the order of the
operators is the reverse of their linear order in the English sentence.

In general, a sentence with $n$ operators (drawn from $\neg$, $\exists$,
$\forall$) will have $n\text{!}$ scope renderings. 

Not all of these interpretations will always be distinct. For example, two
quantifiers of the same kind are scopally commutative:
\begin{enumerate}
  \item
    $\exists \nu.\, \exists u.\, \phi = \exists u.\, \exists
    \nu.\,\phi$
    
  \item
    $\forall \nu.\, \forall u.\, \phi = \forall u.\, \forall
    \nu.\,\phi$
\end{enumerate}


\section{FOPL as semantic metalanguage}

We can use FOPL to regiment our semantic metalanguage.

For example, here is one way to notate a property that holds of an $x$ iff $x$
saw a linguist:
\[\lambda x.\,\exists y.\,\emph{linguist}\,y \wedge \emph{saw}(x, y)\]

Keep this in mind. We'll be seeing more of it in the coming weeks.

\section{Syntax of FOPL}

Vocabulary:
\begin{itemize}
  \item
    \textbf{Terms}: an infinite stock of \textbf{variables}: $x$, $y$, $z$, \dots

  \item
    A collection of $n$-ary \textbf{predicates}: \emph{runs}, \emph{likes},
    \emph{gave}, \dots

  \item
    The propositional logic \textbf{connectives}: $\neg$, $\wedge$, $\vee$,
    $\Ra$. Plus punctuation: ${.}$, $($, and $)$.

  \item
    An \textbf{existential quantifier} $\exists$, and a \textbf{universal
    quantifier} $\forall$.
\end{itemize}

Complex formulas. The WFF of propositional logic is the smallest set such
that:
\begin{itemize}
  \item
    Predicates applied to the appropriate number of terms are in WFF\@. E.g.,
    $\emph{left}\,x$, $\emph{saw}(x,y)$, \dots\ These are the atomic formulas. 

  \item
    If $\phi$ is in WFF, then $\neg\phi$ is in WFF\@.
    
  \item
    If $\phi$ and $\psi$ are in WFF, then $(\phi \wedge \psi)$, $(\phi \vee
    \psi)$, and $(\phi \Ra \psi)$ are all in WFF\@.

  \item
    If $\phi$ is in WFF, then $(\exists \nu.\,\phi)$ and $(\forall \nu.\,\phi)$
    are in WFF, for any variable $\nu$.
\end{itemize}

As before, we adopt the convention of omitting outermost parentheses. I also
like (as above) to omit parentheses when doing so doesn't create ambiguity
(cf.~fn.~\ref{fn}), but whether you do so is up to you.

\section{Semantics of FOPL}

As in propositional logic, we need a way to assign values to variables. This
time, however, the variables denote things of type \texttt{e}:
\begin{itemize}
  \item
    $\sv{v}^g = g\,v$
\end{itemize}

Proper names and predicates will be valued by \sv{\cdot}$^g$ in the way we've
done in class:
\begin{itemize}
  \item
    $\sv{\emph{bob}}^g = \con{b}$

  \item
    $\sv{\emph{left}}^g = \lambda x.\,\con{left}\,x$

  \item
    $\sv{\emph{likes}}^g = \lambda x.\,\lambda y.\,\con{likes}\,(y,x)$
    
  \item
    \dots
\end{itemize}

Predications are evaluated by finding the values of the predicate and its
arguments, and then applying the former to the latter:
\begin{itemize}
  \item
    $\sv{P\,\nu}^g = \sv{P}^g\,\sv{\nu}^g$

  \item
    $\sv{R\,(\nu, u )}^g = \sv{R}^g\,(\sv{\nu}^g, \sv{u }^g)$

  \item
    \dots
\end{itemize}

The meanings of the connectives are unchanged from propositional logic:
\begin{itemize}
  \item
    $\sv{\neg\phi}^g = 1 - \sv{\phi}^g$

  \item
    $\sv{\phi \wedge \psi}^g = \texttt{Min}\,\set{\sv{\phi}^g, \sv{\psi}^g}$

  \item
    $\sv{\phi \vee \psi}^g = \texttt{Max}\,\set{\sv{\phi}^g, \sv{\psi}^g}$

  \item
    $\sv{\phi \Ra \psi}^g = \texttt{Max}\,\set{\sv{\neg\phi}^g, \sv{\psi}^g}$
\end{itemize}

Finally, the meanings of the quantifiers rely on \emph{assignment modification}:
\begin{itemize}
  \item
    $\sv{\exists \nu.\, \phi}^g = \texttt{Max}\,\set{\sv{\phi}^{g[\nu \ra x]} :
    x \in \texttt{e}}$

  \item
    $\sv{\forall \nu.\, \phi}^g = \texttt{Min}\,\set{\sv{\phi}^{g[\nu \ra x]} :
    x \in \texttt{e}}$
\end{itemize}

Relies on \textbf{minimal assignment modification}:
\begin{itemize}
  \item
    $g[\nu \ra x]$ is the assignment $h$ such that $h\,\nu = x$, and for any
    $u  \neq \nu$, $h\,u = g\,u$.
\end{itemize}

Mnemonically, you can think of $g[\nu \ra x]$ as ``the assignment mapping $\nu$
to $x$, but otherwise just like $g$''.

Essentially, existential quantification is like a huge disjunction (if $a$
\textbf{or} $b$ \textbf{or} $c$ \textbf{or} \dots\ makes $\phi$ true, then
$\exists \nu.\,\phi$ is true), and universal quantification is like a huge
conjunction (if $a$ \textbf{and} $b$ \textbf{and} $c$ \textbf{and} \dots\ make
$\phi$ true, then $\forall \nu.\,\phi$ is true).

\end{document}
