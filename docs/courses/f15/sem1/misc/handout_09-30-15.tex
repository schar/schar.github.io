\documentclass{article}
\usepackage{%
  stmaryrd,
  mathtools,
  amsmath,
  amssymb,
  booktabs,
  parskip,
  qtree,
  enumitem
}
\usepackage[silent]{mathspec}
\defaultfontfeatures{Ligatures=TeX}
\setmainfont[%
  Numbers=OldStyle,
  Kerning=Uppercase,
  Scale=MatchLowercase
]{Aldus nova Pro}
\setsansfont[%
  BoldFont={Univers Next Pro Medium},
  Scale=.9
]{Univers Next Pro Regular}
\setmonofont[%
  Scale=MatchLowercase
]{Andale Mono} 
\setmathfont(Digits,Latin,Greek)[%
  Scale=MatchLowercase
]{Neo Euler}
%\scriptspace=0pt 
\usepackage{MnSymbol}
\usepackage[normalem]{ulem}
\usepackage[colorlinks]{hyperref}
\newcommand{\con}[1]{\textsf{\textsc{#1}}}
\newcommand{\sv}[1]{\ensuremath{\lsem #1 \rsem}}
\newcommand{\set}[1]{\ensuremath{\{ #1 \}}}
\newcommand{\ceq}{\coloneqq}
\newcommand{\Ra}{\Rightarrow}
\newcommand{\ra}{\rightarrow}
\newcommand{\bsf}[1]{\textbf{\textsf{#1}}}
\newcommand{\ab}[1]{\ensuremath{\langle #1 \rangle}}
\newcommand{\AB}[1]{\ensuremath{(#1)}}
\newcommand{\wff}{\textsc{\textsf{wff}}}
\usepackage[margin=1.4in]{geometry}
\setlength{\parindent}{0pt}
\author{%
  Simon Charlow
  (\href{mailto:simon.charlow@rutgers.edu}%
  {{simon.charlow@rutgers.edu}})
}
\title{\textbf{Type theory}}
\date{September 30, 2015}
\synctex=1
\begin{document}\maketitle

\section{Lambda calculus, redux}

Why we bother:
\begin{itemize}
  \item
    $\lambda$s let us build functions \textbf{anonymously}.

  \item
    Simplification of $\lambda$-terms is a straightforward way to compute values
    for complex expressions.
\end{itemize}

\subsection{Syntax}

Syntax: the set of $\lambda$-terms, $\Lambda$, is the smallest set meeting the
following conditions:%
\footnote{%
  As we saw in the last class, in BNF this amounts to $E \Coloneqq V \mid
  (\lambda V.\,E) \mid (E_1 \, E_2)$.
}
\begin{enumerate}
  \item
    (\textbf{Variables})
    If $x$ is a variable, $x \in \Lambda$.

  \item
    (\textbf{Abstraction})
    If $x$ is a variable and $M \in \Lambda$, then $(\lambda x.\,M) \in
    \Lambda$.

  \item
    (\textbf{Application})
    If $M,N \in \Lambda$, then $(M\,N) \in \Lambda$.
\end{enumerate}

\textbf{And that is literally it.} But some abbreviatory conventions improve
usability: 
\begin{enumerate}
  \item
    Application is left-associative. For example:
    \begin{itemize}
      \item
        `$M\, N\, L$' instead of `$((M\, N)\, L)$'
    \end{itemize}
    
  \item
    Parentheses can be omitted whenever doing so doesn't create ambiguity. For
    example:
    \begin{itemize}
      \item
         `$\lambda x.\,M\,N$' instead of `$\lambda x.\,(M\,N)$'
         
      \item
        `$\lambda x.\,M$' instead of `$(\lambda x.\,M)$'

      \item
        `$\lambda x.\,\lambda y.\,M$' instead of `$(\lambda x.\,(\lambda
        y.\,M))$'
    \end{itemize}
    NB\@: $\lambda x.\, M\, y$ is equivalent to $\lambda x.\, (M\, y)$ and
    $(\lambda x.\,(M\, y))$, but \emph{not} $(\lambda x.\,M)\, y$. I.e., the
    scope of a $\lambda$ always extends as far to the right as possible
    (i.e.~while still yielding a well-formed term).

  \item
    A sequence of abstractions can be abbreviated. For example:
    \begin{itemize}
      \item 
        `$\lambda xyz.\,M$' instead of `$\lambda x.\,\lambda y.\,\lambda
        z.\,M$'
    \end{itemize}
\end{enumerate}

I recommend mastering at least the first two conventions. Whether you use the
third is up to you\dots

\subsection{``Semantics''} 

Core equivalences:
\begin{itemize}
  \item
    (\textbf{Renaming bound variables}) 
    $\lambda x.\,M \underset{\alpha}{\longrightarrow} \lambda y.\, M[x \ra y]$

  \item
    (\textbf{Applying functions to arguments})
    $(\lambda x.\,M)\,y \underset{\beta}{\longrightarrow} M[x \ra y]$

  \item
    (\textbf{Extensionality}) $f \underset{\eta}{\longrightarrow} \lambda
    x.\, f\, x$
\end{itemize}

Two of these equivalences rely on the notion of \textbf{substitution}, defined
as follows: 
\begin{align}
  \notag
  M[x \ra y] \text{~is the formula you get by replacing all $M$'s \textbf{free}
  occurrences of $x$ with $y$.}
\end{align}
%
The \textbf{free} variables in an expression $E$ (`$\con{FV}\,E$') are the
variables in $E$ not \textbf{bound} by a $\lambda$:
\begin{align}
  \notag
  \begin{array}{r@{}c@{}l}
    \con{FV}\,x & {}={} & \set{x}
    \\[.35em] 
    \con{FV}\,(\lambda x.\,M) & {}={} & \con{FV}\,M - \set{x}
    \\[.35em]  
    \con{FV}\,(M\,N) & {}={} & \con{FV}\,M \cup \con{FV}\,N
  \end{array}
\end{align}
%
In other words, you can read $M[x \ra y]$ as ``$M$, but where the (free) $x$'s
becomes $y$'s''.

Some examples of substitution in action: 
\begin{align}
  \notag
  x[x \ra y] &= y
  \\\notag
  (\lambda x.\,x)[x \ra y] &= \lambda x.\,x
  \\\notag
  (\lambda f.\,f\,x)[x \ra y] &= \lambda f.\, f\, y
\end{align}

Substition has various notations. Sometimes you see `$M[x \ceq y]$'. Sometimes,
as in our last class, you see `$M[y\mathrel/x]$'. I'm hoping the present choice
will be the most perspicuous. Let me know.

A couple examples of $\alpha$, $\beta$, and $\eta$ in action (notice that
$\alpha$-, $\beta$-, and $\eta$- equivalence can apply to proper
\emph{sub-formulas} of any $\lambda$-expression, as well as maximal
$\lambda$-expressions):
\begin{align}
  \begin{array}{c}
    \notag
    \lambda f.\,\lambda x.\, f\,x \underset{\alpha}{\longrightarrow} \lambda
    f.\,\lambda z.\,f\,z \underset{\alpha}{\longrightarrow} \lambda g.\, \lambda
    z.\, g\,z
    \\[1em]
    (\lambda f.\, \lambda x.\, f\,x)\,(\lambda y.\, y)
    \underset{\beta}{\longrightarrow} \lambda x.\, (\lambda y.\,y)\,x
    \underset{\beta}{\longrightarrow} \lambda x.\, x
    \\[1em] 
    \lambda f.\, f \underset{\eta}{\longrightarrow} \lambda f.\, \lambda x.\,
    f\, x \underset{\eta}{\longrightarrow} \lambda f.\, \lambda x.\, \lambda
    y.\, f\, x\, y
  \end{array}
\end{align}
%
A $\beta$-reduction is \textbf{invalid} if it places a variable within the scope
of a co-indexed $\lambda$:
\begin{align}
  \notag
  (\lambda x.\, \lambda y.\, f\,x\,y)\,y
  \underset{\mathclap{\beta\text{?~}\textbf{NO.}}}{\longrightarrow}
  \lambda y.\, f\, y\, y
\end{align}
%
To avoid this unfortunate situation, exploit $\alpha$-equivalence:
\begin{align}
  \notag
  (\lambda x.\, \lambda y.\, f\,x\,y)\,y
  \underset{\alpha}{\longrightarrow}
  (\lambda x.\, \lambda z.\, f\,x\,z)\,y
  \underset{\alpha}{\longrightarrow}
  \lambda z.\, f\, y\, z
\end{align}

An $\alpha$-equivalence is \textbf{invalid} for an expression $E$ if it binds a
free variable in $E$:
\begin{align}
  \notag
  \lambda x.\,x\,y
  \underset{\mathclap{\alpha\text{?~}\textbf{NO.}}}{\longrightarrow}
  \lambda y.\,y\, y
\end{align}

To avoid this unfortunate situation\dots\ Don't get into it!

\section{Types}

The variety of the $\lambda$-calculus defined in the last section is
\textbf{untyped}. Because of this, you can define functions like the following:
\begin{align}
  \notag
  \omega \ceq \lambda x.\, x\, x
\end{align}
%
Which you can use to define the following:
\begin{align}
  \notag
  \Omega \ceq \omega \, \omega
\end{align}
%
As we saw in the last class, $\Omega$ has an interesting property:
$\beta$-``reducing'' it gives you exactly what you already had!!
\begin{align}
  \notag
  \Omega \underset{\beta}{\longrightarrow} \Omega
\end{align}
%
We'll be working with a different system in this class, called the
\textbf{simply-typed} $\lambda$-calculus.

What are \textbf{types}? Pretty simple: types are nothing more than \textbf{sets
of objects}! For example, the type $\mathbb{N}$ is the set of natural numbers. 

In semantics, we work with two basic types: the type of individuals (roughly,
things that can go in subject or object position%
\footnote{%
  NB\@: semanticists think of lots of things as individuals, including numbers,
  tables and chairs, pets, bacteria, and so on. 
}), and the type of truth values. By convention, the type of individuals is
written `$\texttt{e}$' (for `entity', probably), and the type of truth values is
written `$\texttt{t}$'.
\begin{align}
  \notag
  \texttt{e} & \Coloneqq \set{x : x \text{~is an individual}}
  \\\notag
  \texttt{t} & \Coloneqq \set{0, 1}
\end{align}
%
Using our basic types, we can give an inductive definition for the set of
\textbf{all types} $\tau$, as follows:
\begin{enumerate}
  \item
    $\texttt{e}, \texttt{t} \in \tau$.

  \item
    If $\alpha, \beta \in \tau$, then $\ab{\alpha, \beta} \in \tau$.

  \item
    Nothing else is in $\tau$.
\end{enumerate}

Some examples of types according to this ``grammar'': 
\begin{itemize}
  \item
    \texttt{e}
    
  \item
    \ab{\texttt{e}, \texttt{t}}

  \item 
    \ab{\ab{\texttt{e}, \texttt{t}}, \texttt{t}}

  \item
    \ab{\ab{\texttt{e},\texttt{e}}, \ab{\ab{\texttt{e},\texttt{e}},
    \ab{\texttt{e},\texttt{e}}}}
\end{itemize}

The simple types are the individuals and the truth values. What are the complex
types? The types of \textbf{functions}: \ab{\alpha, \beta} is the type of
functions from things of type $\alpha$ to things of type $\beta$.

An example. Let's begin with the set of linguists. This set has a characteristic
function, which we might name $\con{linguist}$. This function can be represented
in $\lambda$-notation as follows (notice that this is just the $\eta$-expansion
of \con{linguist}!):
\begin{align}
  \notag
  \lambda x.\, \con{linguist}\, x
\end{align}
%
This function
takes individuals as arguments and returns truth values.  Accordingly, its type
is \ab{\texttt{e}, \texttt{t}}. 

Likewise, an \ab{\ab{\texttt{e},\texttt{t}},\texttt{t}} function takes functions
from individuals to truth values as arguments, and returns truth values. A
trivial example is the following:
\begin{align}
  \notag
  \lambda P.\, 1
\end{align}

In the simply-typed $\lambda$-calculus, every function is \textbf{partial}. That
is, every function expects its arguments to be of a certain type, and doesn't
know what to do otherwise. For example, $\lambda x.\, \con{linguist}\, x$
wouldn't know what to do if you fed it $\lambda x.\, \con{philosopher}\, x$. The
result would be \textbf{undefined}.

There are a number of ways to notate restrictions on types. Heim \& Kratzer use
the following convention (more generally, they use the `:' notation to
indicate partiality):
\begin{align}
  \notag
  \lambda x : x \in \texttt{e}.\, \con{linguist}\, x
\end{align}
%
Another way to go (the way I prefer, maybe because it saves symbols), is to use
subscripts:
\begin{align}
  \notag
  \lambda x_{\texttt{e}}.\, \con{linguist}\, x
\end{align}
%
In any case, type notations are often omitted when context makes clear what an
expression's type is.

Some other examples:
\begin{itemize}
  \item
    $\sv{\text{blue}} = \lambda x_{\texttt{e}}.\, \con{blue}\, x$

  \item
    $\sv{\text{devoured}} = \lambda x_{\texttt{e}}.\, \lambda y_{\texttt{e}}.\,
    \con{devoured}\, x\, y$

  \item
    $\sv{\text{not}} = \lambda p_{\texttt{t}}.\, 1 - p$

  \item
    $\sv{\text{and}} = \lambda p_{\texttt{t}}.\, \lambda q_{\texttt{t}}.\, p = q
    = 1$
\end{itemize}

In order:
\begin{itemize}
  \item
    The type of a simple adjective is a function from individuals to truth
    values, type \ab{\texttt{e}, \texttt{t}}. 

  \item
    The type of a transitive verb is a function from individuals to: a function
    from individuals to truth values, type \ab{\texttt{e}, \ab{\texttt{e},
    \texttt{t}}}. 

  \item
    The type of sentential negation is a function from truth values to truth
    values, type \ab{\texttt{t}, \texttt{t}}.
    
  \item
    The type of sentential conjunction is a function from truth values to: a
    function from truth values to truth values, type \ab{\texttt{t},
    \ab{\texttt{t}, \texttt{t}}}.
\end{itemize}

In this manner of speaking, we can talk about denotations having types, as well
as syntactic units (i.e., the type associated with a piece of syntax's
denotation). We can use the following notation to indicate that an expression
has a certain type:
\begin{align}
  \notag
  \text{devoured} :: \ab{e, \ab{e, t}}
\end{align}

At the risk of burdening you with another notational convention, I'll mention
that the way I prefer to write types in my own work (because it's way easier to
parse than the brackets) is using arrows:%
\footnote{%
  In fact, you could \emph{omit} the arrows entirely and still have a perfectly
  good notation scheme for types. But that's getting a bit telegraphic for most
  people's tastes.
}
\begin{align}
  \notag
  \lfloor e \rfloor &= e
  \\\notag
  \lfloor e \ra t \rfloor &= e \ra t
  \\\notag
  \lfloor\ab{e, \ab{e, t}}\rfloor &= e \ra e \ra t
  \\\notag
  \lfloor\ab{\ab{e, t}, t}\rfloor &= (e \ra t) \ra t
\end{align}
%
Notice that in the simply-typed $\lambda$-calculus, it \textbf{makes no sense}
for a function to apply to itself:
\begin{itemize}
  \item
    Take an arbitrary $f$. Since $f$ is a function, it will have type
    \ab{\alpha, \beta}, for some types $\alpha$ and $\beta$. 

  \item 
    But then, for $f\,f$ to be defined, $f$ needs to be able to take itself as
    an argument. 

  \item 
    So then the functor $f$'s type must be \ab{\ab{\alpha, \beta}, \beta}. 
    
  \item
    But then the argument $f$'s type must also be \ab{\ab{\alpha, \beta},
    \beta}.
    
  \item
    In which case $f$'s type must actually be \ab{\ab{\ab{\alpha, \beta},
    \beta}, \beta}. 
    
  \item
    And so on\dots
\end{itemize}

This means that $\omega$, which crucially involves self-application, is
impossible to define in a simply-typed system. Ergo, $\Omega$ is impossible to
define. 

In fact, every simply-typed $\lambda$-term is guaranteed to \textbf{terminate}
into a unique \textbf{normal form}, in which no further $\beta$-reductions are
possible. Because every expression has a \emph{unique} normal form, the order of
reductions never makes any difference. For example:
\begin{align}
  \begin{array}{r@{}c@{}l}
    (\lambda x.\, f\,x) \, ((\lambda y.\, g\, y)\, z)
    \underset{\beta}\longrightarrow{}
    &
    (\lambda x.\, f\,x) \, (g\,z)
    &
    {}\underset{\beta}\longrightarrow f\, (g\, z)
    \\
    (\lambda x.\, f\,x) \, ((\lambda y.\, g\, y)\, z)
    \underset{\beta}\longrightarrow{}
    &
    f\,((\lambda y.\, g\, y)\, z)
    &
    {}\underset{\beta}\longrightarrow  
    f\, (g\, z)
  \end{array}
\end{align}
%
%(Notice that I am using subscripts here both to indicate the type of a
%$\lambda$-expression's arguments, and also the types of the epxressions in its
%body.)
Unlike $\omega$, it is possible to assign types to all the expressions in this
term in a way that achieves a defined end result. Exercise: come up with one
such typing scheme. 

\end{document}
