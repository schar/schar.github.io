%\begin{preamble}
\documentclass[twocolumn]{article}
\setlength{\columnsep}{2.5em}
\usepackage{amsthm}
\usepackage{amsmath}
\usepackage{mathtools}
\usepackage{amssymb}
\usepackage{tikz-qtree}
\usepackage{stmaryrd}
\newcommand{\sv}[1]{\ensuremath{\llbracket #1 \rrbracket}}
\newcommand{\svv}[1]{\ensuremath{\left\llbracket #1 \right\rrbracket}}
\newcommand{\AB}[1]{\ensuremath{\langle #1 \rangle}}
\newcommand{\bsf}[1]{\textsf{\bfseries #1}}
\newcommand{\rex}[1]{(\ref{#1})}
\newcommand{\ra}{\ensuremath{\rightarrow}}
\newcommand{\Ra}{\ensuremath{\Rightarrow}}
\newcommand{\nRa}{\ensuremath{\nRightarrow}}
\newcommand{\ceq}{\ensuremath{\coloneqq}}
\usepackage[landscape,margin=.6in]{geometry}
\usepackage[normalem]{ulem}
\usepackage{hyperref}
\usepackage{tikz}
\usepackage{enumitem}
\setlist[itemize]{leftmargin=0pt}
\def\labelitemi{} 
\usepackage{gb4e}
\newcommand\bex{\begin{exe}}
\newcommand\eex{\end{exe}}
\newcommand\bta{\begin{tabular}{c}}
\newcommand\etab{\end{tabular}}
\newcommand\bfi{\begin{figure}}
\newcommand\efi{\end{figure}}
\newcommand\btr{\begin{tikzpicture}[scale=.9]\tikzset{level distance=20pt,sibling distance=0pt}}
\newcommand\etr{\end{tikzpicture}}
%\end{preamble}
\begin{document}
\begin{itemize}
\item\noindent\textbf{\Large October 7--10: LF and QR}


\section{Semantics for topicalization}

\item Recall an example from last week. As we remarked then, the following is unambiguously associated with an interpretation on which questions don't vary with answerers (in contrast with \emph{everybody answered at least one question}): %
\bex
	\ex At least one question, everybody answered.
\eex

\item To analyze this case, we'll assume that the relationship between a topicalized expression and an extraction gap is essentially the same as the relationship between a relative pronoun like \textsf{who} and an extraction gap.%

\item In other words, topicalization movement leaves a trace and a co-indexed, c-commanding abstraction operator.%

\item We'll begin with a simpler case with just one quantifier before moving onto an example with two quantifiers:%
\begin{figure}[!ht]
	\[\begin{array}{c}
	\textsf{%
		\bta\btr
		\Tree [.TopP \node(obj){at least one question}; [.$\Lambda$ 3 [.S Bill [.VP answered \node(t){$t_3$}; ] ] ] ]%
		\draw[->] (t)..controls +(south west:2) and +(south:2)..(obj);
		\etr\etab
	}
		\\
		\begin{array}{r@{}ll}
			\sv{\textsf{TopP}}^g &{}= \sv{\textsf{at least one question}}^g(\sv{\Lambda}^g) & \bsf{FA}%
			\\
			&{}= \text{aloq}'(\sv{\Lambda}^g) & \text{Lexicon}
			\\
			&{}= \text{aloq}' (\lambda x.\sv{\textsf{S}}^{g[x/3]}) & \bsf{PA}
			\\
			&{}= \text{aloq}' (\lambda x.\sv{\textsf{Bill}}^{g[x/3]}(\sv{\textsf{VP}}^{g[x/3]})) & \bsf{FA}%
			\\
			&{}= \text{aloq}' (\lambda x.\sv{\textsf{VP}}^{g[x/3]})(\text{b}) & \text{Lexicon}%
			\\
			&{}= \text{aloq}' (\lambda x.\sv{\textsf{answered}}^{g[x/3]}(\sv{t_{\textsf{3}}}^{g[x/3]})(\text{b})) & \bsf{FA}%
			\\
			&{}= \text{aloq}' (\lambda x.\text{answered}'(\sv{t_{\textsf{3}}}^{g[x/3]})(\text{b})) & \text{Lexicon}%
			\\
			&{}= \text{aloq}' (\lambda x.\text{answered}'(x)(\text{b})) & \text{Pronoun rule}%
		\end{array}
	\end{array}\]
	\caption{\textsf{at least one question, Bill answered}}
	\label{fig1a}
\end{figure}

\item If $\text{aloq}' = \lambda P.\,\{x : \text{question}'(x)\} \cap \{x : P(x)\} \neq \emptyset$, then the result $\beta$-reduces to the following truth condition:%
\[\{x : \text{question}'(x)\} \cap \{x : \text{answered}'(x)(\text{b})\} \neq \emptyset\]

\item Now for an example with two quantifiers. This works the same as the first case. The only difference is the presence of an additional quantifier. But since that extra quantifier is in subject position, interpretation proceeds seamlessly.%
\begin{figure}[!ht]
	\[\begin{array}{c}
	\textsf{%
		\bta\btr
		\Tree [.TopP \node(obj){at least one question}; [.$\Lambda$ 3 [.S everybody [.VP answered \node(t){$t_3$}; ] ] ] ]%
		\draw[->] (t)..controls +(south west:2) and +(south:2)..(obj);
		\etr\etab
	}
		\\
		\begin{array}{r@{}ll}
			\sv{\textsf{TopP}}^g &{}= \sv{\textsf{at least one question}}^g(\sv{\Lambda}^g) & \bsf{FA}%
			\\
			&{}= \text{aloq}'(\sv{\Lambda}^g) & \text{Lexicon}
			\\
			&{}= \text{aloq}' (\lambda x.\sv{\textsf{S}}^{g[x/3]}) & \bsf{PA}
			\\
			&{}= \text{aloq}' (\lambda x.\sv{\textsf{everybody}}^{g[x/3]}(\sv{\textsf{VP}}^{g[x/3]})) & \bsf{FA}%
			\\
			&{}= \text{aloq}' (\lambda x.\text{eb}'(\sv{\textsf{VP}}^{g[x/3]})) & \text{Lexicon}%
			\\
			&{}= \text{aloq}' (\lambda x.\text{eb}'(\sv{\textsf{answered}}^{g[x/3]}(\sv{t_{\textsf{3}}}^{g[x/3]}))) & \bsf{FA}%
			\\
			&{}= \text{aloq}' (\lambda x.\text{eb}'(\text{answered}'(\sv{t_{\textsf{3}}}^{g[x/3]}))) & \text{Lexicon}%
			\\
			&{}= \text{aloq}' (\lambda x.\text{eb}'(\text{answered}'(x))) & \text{Pronoun rule}%
		\end{array}
	\end{array}\]
	\caption{\textsf{at least one question, everybody answered}}
	\label{fig1}
\end{figure}

\item Notice that this structure is unambiguously associated with the interpretation on which the set of questions and the set of things everybody answered have a nonempty intersection: %
\[\{x : \text{question}'(x)\} \cap \{x : \text{ppl}' \subseteq \{y : \text{answered}'(x)(y)\}\} \neq \emptyset\]%

\item Specifically, the property that characterizes things everybody answered is asserted to be a member of $\sv{\textsf{at least one question}}^g$. This holds iff the set of things everybody answered and the set of questions have a non-empty intersection.%

\item The overall theme: movement creates a structure such that the sister of the moved quantifier denotes a property (type \AB{e,t}), exactly the right sort of thing to combine with the moved quantifier (type \AB{\AB{e,t},t}) by functional application. %

\item It is important that the trace is type $e$. What happens if the trace is type \AB{\AB{e,t},t} and the type of $\Lambda$ is \AB{\AB{\AB{e,t},t},t}?%


\section{Things quantifiers can do}

\item Quantifiers as internal arguments of transitive verbs:
\bex
	\ex John knows$_{\AB{e,\AB{e,t}}}$ [many linguists]$_{\AB{\AB{e,t},t}}$
	\ex Uni licked$_{\AB{e,\AB{e,t}}}$ [every cat]$_{\AB{\AB{e,t},t}}$
\eex

\item Quantifiers as internal arguments of ditransitive verbs and ``transitive VPs'': %
\bex
	\ex Uni showed$_{\AB{e,\AB{e,\AB{e,t}}}}$ [no cat]$_{\AB{\AB{e,t},t}}$ Porky
	\ex Mary read$_{\AB{e,\AB{e,\AB{e,t}}}}$ [every child]$_{\AB{\AB{e,t},t}}$ [a story]$_{\AB{\AB{e,t},t}}$%
\eex

\item Ambiguity in cases of two quantifiers: 
\bex
	\ex A doctor examined every patient. 
	\ex A sentry guards every building.
	\ex A teacher gave a book to every child. 
\eex

\item Binding:
\bex
	\ex Every boy$_i$ likes his$_i$ mother.
	\ex We will sell no wine$_i$ before its$_i$ time. 
	\ex A member of each committee$_i$ voted to abolish it$_i$
\eex


\section{LF}

\item Our account of these phenomena will closely parallel the account of topicalization we began with. It'll require two substantial shifts in our theory:% 
\begin{enumerate}
	\item \textbf{LF} (mnemonic for ``logical form'', but really means something distinct from what is usually meant by that term): an abstract level of representation which serves as the input to $\sv{\cdot}^g$. %
	\item \textbf{QR} (abbreviation for ``quantifier raising''): a movement operation that may occur between surface structures and LF. Results when a quantificational DP is moved, leaving behind a trace and inserting an abstraction index co-indexed with the trace. %
\end{enumerate}

\item ``Y-model'' of syntax (we have not talked so much about D-structure or PF):
\[\bta\btr
	\tikzset{every tree node/.style={align=center,anchor=north}}
	\Tree [.D-structure [.S-structure {LF\\Semantics} {PF\\Pronunciation} ] ]
\etr\etab\]

\item So the relationship between overt structure and meaning is less direct than we had been assuming (hoping?). As we will see, the interpreted structure can be a good deal more abstract than what we see on the surface. %

\item What we will assume is that the interpreted structures with quantifiers in them can look a lot like topicalized cases. %

\item Like topicalization, QR inserts a trace and a co-indexed, c-commanding abstraction operator. The sister of the moved quantifier will have type \AB{e,t},  the right sort of thing to combine with the quantifier by functional application. %
%\item Characterizing QR formally (important things: trace is lower type---think about what happens if trace is higher type; index on trace is co-indexed with inserted abstraction index).%

%\item This means, in each case, the quantifier is adjoined to something that characterizes a property. %



\subsection{Examples (take notes!)}

\item In situ quantification:
\[\bta\btr
	\Tree [.S \node(obj){every cat}; [.$\Lambda$ 1 [.S Uni [.VP licked \node(t){$t_1$}; ] ] ] ]%
	\draw[->] (t)..controls +(south west:1.5) and +(south:2)..(obj);
\etr\\
$\sv{\textsf{every cat}}^g(\lambda x.\,\text{licked}'(x)(\text{u}))$
\etab\]

\item In situ quantification with ditransitives:
\[\bta\btr
	\Tree [.S \node(obj){no cat}; [.$\Lambda$ 1 [.S Uni [.VP [.VP showed \node(t){$t_1$}; ] Porky ] ] ] ]%
	\draw[->] (t)..controls +(south west:2) and +(south:2)..(obj);
\etr\\
$\sv{\textsf{no cat}}^g(\lambda x.\,\text{showed}'(x)(\text{p})(\text{u}))$
\etab\]



\item Inverse scope:
\[\bta\btr
	\Tree [.S \node(obj){every patient}; [.$\Lambda$ 1 [.S {a doctor} [.VP examined \node(t){$t_1$}; ] ] ] ]%
	\draw[->] (t)..controls +(south west:2) and +(south:2)..(obj);
\etr\\
$\sv{\textsf{every patient}}^g(\lambda x.\sv{\textsf{a doctor}}^g(\text{examined}'(x)))$
\etab\]


\item Restoring surface scope:
\[\bta\btr
	\Tree [.S \node(subj){a doctor}; [.$\Lambda$ 2 [.S \node(obj){every patient}; [.$\Lambda$ 1 [.S \node(t2){$t_2$}; [.VP examined \node(t1){$t_1$}; ] ] ] ] ] ]%
	\draw[->] (t1)..controls +(south west:2) and +(south:2)..(obj);
	\draw[->] (t2)..controls +(south west:2) and +(south:2)..(subj);
\etr\\
$\sv{\textsf{a doctor}}^g(\lambda y.\sv{\textsf{every patient}}^g(\lambda x.\text{examined}'(x)(y)))$
\etab\]


\item Binding:
\[\bta\btr
	\Tree [.S \node(subj){every boy}; [.$\Lambda$ 1 [.S \node(t){$t_1$}; [.VP likes [.DP his$_i$ mom ] ] ] ] ]%
	\draw[->] (t)..controls +(west:1) and +(south:2)..(subj);
\etr\\
$\sv{\textsf{every boy}}^g(\lambda x.\,\text{likes}'(\text{mom}'(x))(x))$
\etab\]

\item Inverse linking:
\[\bta\btr
	\Tree [.S \node(qp){each committee}; [.$\Lambda$ 7 [.S [.DP a [.NP member [.PP of \node(t){$t_7$}; ] ] ] [.VP {voted to abolish} it$_7$ ] ] ] ]%
	\draw[->] (t)..controls +(south west:2) and +(south:2)..(qp);
\etr\etab\]

\item A quantifier ${\cal Q}_1$ can only bind into a quantifier ${\cal Q}_2$ if ${\cal Q}_1$ has scope over ${\cal Q}_2$, where ${\cal Q}_1$ has scope over ${\cal Q}_2$ iff ${\cal Q}_1$ c-commands ${\cal Q}_2$ at LF:%
\[\bta\btr
	\Tree [.S \node(subj){a sentry}; [.$\Lambda$ 1 [.S [.DP \edge[roof]; \node(obj){every building in his$_1$ charge}; ] [.$\Lambda$ 2 [.S \node(t1){$t_1$}; [.VP guards \node(t2){$t_2$}; ] ] ] ] ] ]%
	\draw[->] (t1)..controls +(west:3) and +(south:5)..(subj);
	\draw[->] (t2)..controls +(south west:2) and +(south:2)..(obj);
\etr\etab\]

\item If we try to assign the object quantifier scope over the subject, we inevitably end up unbinding the pronoun \textsf{his}$_1$:%
\[\bta\btr
	\Tree [.S [.DP \edge[roof]; \node(obj){every building in his$_1$ charge}; ] [.$\Lambda$ 2 [.S \node(subj){a sentry}; [.$\Lambda$ 1 [.S \node(t1){$t_1$}; [.VP guards \node(t2){$t_2$}; ] ] ] ] ] ]%
	\draw[->] (t1)..controls +(west:1)..(subj);
	\draw[->] (t2)..controls +(south west:1) and +(south:3)..(obj);
\etr\etab\]

\item Over-generation concern: unlike pronouns, QR should never be able to un-bind traces. That is, the following LF shouldn't be allowed: %
\[\bta\btr
	\Tree [.S [.DP a [.NP member [.PP of \node(t){$t_7$}; ] ] ] [.$\Lambda$ 2 [.S \node(qp){each committee}; [.$\Lambda$ 7 [.S $t_2$ [.VP {voted to abolish} it$_7$ ] ] ] ] ] ]%
	\draw[->] (t)..controls +(south east:1) and +(south:1)..(qp);
\etr\etab\]

\item The truth conditions derived in this case are bizarre. Given an assignment $g$, they require there to be a member of $g(7)$ such that, for each committee $x$, s/he voted to abolish $x$. %

\item Obviously, this isn't a possible reading of the sentence, and so the rule that relates S-structure with LFs will have to be carefully formulated so avoid this sort of outcome.%

%\item HK: relationship between NL and first-order logic. Issues: quantifiers combine with properties (sentences), quantifiers like most, expanding the lexicon of FOL%
%\item Good time to spend some time on FOL


\section{Flexible types}

\item There exist other ways to interpret object quantifiers in situ. For example, we might imagine that there is a silent morpheme that either applies to transitive verbs or quantifiers and allows them to compose directly:%
\[\sv{\textsc{sat}_\emptyset}^g = \lambda R_{\AB{e,\AB{e,t}}}.\lambda {\cal Q}_{\AB{\AB{e,t},t}}.\lambda x_e.{\cal Q}(\lambda y.R(y)(x))\]%
\[\sv{\textsc{sat}_\emptyset}^g = \lambda {\cal Q}_{\AB{\AB{e,t},t}}.\lambda R_{\AB{e,\AB{e,t}}}.\lambda x_e.{\cal Q}(\lambda y.R(y)(x))\]%

\item Alternatively, it could be that everything is type \AB{\AB{e,t},t} and transitive verbs are born with a higher type:%
\[\sv{\textsf{John}}^g = \lambda P_{\AB{e,t}}.P(\text{j})~~~~~~~~~~\sv{\textsf{met}}^g = \lambda {\cal Q}_{\AB{\AB{e,t},t}}.\lambda x.{\cal Q}(\lambda y.\text{met}'(y)(x))\]%

\item Yet another possibility is that some expressions are born as type $e$ but shift into type-\AB{\AB{e,t},t} expressions via the following silent morpheme: %
\[\sv{\textsc{lift}}^g = \lambda x_e.\lambda P_{\AB{e,t}}.P(x)\]

\item And we haven't even come close to exploring the entire logical space of possibilities. For example, it might be the case that functional application is not the sole saturative mode of combination(!).%

%\item Here, HK also discuss generalized conjunction (maybe also negation, i.e. not everything/Bill and VP negation)% 



\end{itemize}
\end{document}