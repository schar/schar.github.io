%\begin{preamble}
\documentclass[twocolumn]{article}
\usepackage{amsthm}
\usepackage{amsmath}
\usepackage{mathtools}
\usepackage{amssymb}
\usepackage{example}
\usepackage{qtree}
\usepackage{stmaryrd}
\newcommand{\sv}[1]{\ensuremath{\llbracket #1 \rrbracket}}
\newcommand{\AB}[1]{\ensuremath{\langle #1 \rangle}}
\newcommand{\bsf}[1]{\textsf{\bfseries #1}}
\newcommand{\ceq}{\ensuremath{\coloneqq}}
\newcommand{\ra}{\ensuremath{\rightarrow}}
\usepackage[margin=.75in,landscape]{geometry}
\usepackage[normalem]{ulem}
\usepackage{hyperref}
%\end{preamble}
\begin{document}

\noindent\textbf{\Large Homework for Friday October 10, 2014}


\section{More practice with $\lambda$s}

\begin{itemize}
	\item Reduce the following as much as possible. Show and justify each step in your calculation. %
	\[(\lambda m.\lambda n.m(\lambda f.n(\lambda x.f(x))))(\lambda k.k(\text{left}'))(\lambda k.k(x))\]%
\end{itemize}


\section{Relative clauses}

\begin{itemize}	
	\item Here's a possible tree for the sentence \textsf{the cat Uni likes hates her}.  %
	\[\begin{array}{c}
		\textsf{\Tree [ [ the [ cat [1 [ Uni [ likes $t_1$ ] ] ] ] ] !\qsetw{10em} [ hates her$_1$ ] ]}%
	\end{array}\]
	\begin{itemize}
		\item[$\rhd$] Assign this tree an interpretation relative to an arbitrary assignment function $g$. (It's probably best to go top-down like we did in class.) %
		\item[$\rhd$] Is the trace $t_\textsf{1}$ bound or free in this tree (that is: does the choice of assignment function matter for its interpretation)? Is the object pronoun \textsf{her}$_\textsf{1}$ bound or free in this tree? %
		\item[$\rhd$] What does this tell you about what needs to hold for a variable to get bound by an abstraction index?%
	\end{itemize}
	
	\item Here's a possible tree for the bigger relative clause in the DP \textsf{the man who saw the cat who meowed}. %
	\textsf{\[\begin{array}{c}
		\Tree [ 1 [ $t_1$ [saw [ the [ cat [ who [ 1 [ $t_1$ meowed ] ] ] ] ] ] ] ]%
	\end{array}\]}
	\begin{itemize}
		\item[$\rhd$] Assign this tree an interpretation relative to an arbitrary assignment function $g$. (It's probably best to go top-down like we did in class.) %
		\item[$\rhd$] Which abstraction index binds which trace?%
		\item[$\rhd$] What does this tell you about what needs to hold for a variable to get bound by an abstraction index?%
	\end{itemize}
	
	%\item Over-generation? CK?
\end{itemize}


\section{Quantifiers}

\begin{itemize}
	\item Give type-\AB{\AB{e,t},t} meanings for the following quantificational DPs. Feel free to mix set and function talk, but be explicit about it. Don't worry about the internal composition of the ``determiners''.%
	\begin{enumerate}
		\item not every phonologist
		\item three out of four dentists
		\item every linguist except John
		\item at least four but no more than ten hotels
		\item more than ten or fewer than five semanticists
	\end{enumerate}
	
	\item Derive a meaning for the sentence \textsf{the dog every linguist knows skijors}.%
	
	\item We saw that trying to assign type $e$ meanings to quantificational DPs was destined to fail. But what about vice versa? Can you think of a way to assign type \AB{\AB{e,t},t} meanings to proper names like \textsf{New Jersey} and definite descriptions like \textsf{the Queen of England}?%
	
%	\item What does your entry predict about the truth of \textsf{every unicorn came, and no unicorn came} in a world (like ours?) where unicorns fail to exist, i.e.~where $\sv{\textsf{unicorn}}^g = \emptyset$. What do you think of that?%
	
	\item We saw in class that there was no way to compose a transitive verb (type \AB{e,\AB{e,t}}) with a quantificational DP (type \AB{\AB{e,t},t}). Here is one possible (though partial) fix. Assume the following silent morheme is freely available:  %
	\[\sv{\textsf{BLA}_\emptyset}^g \ceq \lambda R_{\AB{e,\AB{e,t}}}.\lambda {\cal Q}_{\AB{\AB{e,t},t}}.\lambda x_e.{\cal Q}(\lambda y.R(y)(x))\]%
	\begin{itemize}
		\item[$\rhd$] Use \textsf{BLA}$_\emptyset$ to derive a meaning for the ambiguous sentence \textsf{somebody likes everybody}. Be clear about what truth conditions you've derived. Do you see any way to derive the other reading using \textsf{BLA}$_\emptyset$? %
	\end{itemize}
\end{itemize}

\section{Bonus (not required)}

\begin{itemize}
	\item Devise a silent lexical item, perhaps along the lines of \textsf{BLA}$_\emptyset$, that allows you to assign the other interpretation to \emph{somebody likes everybody}.% 
\end{itemize}

\end{document}